\reviewer{Reviewer 3}

\begin{generalcomment}
This paper presents COFFER, a scalable and efficient Trusted Execution Environment (TEE) for commodity RISC-V platforms. It introduces Lightweight PMP Virtualization (LPMP) to overcome hardware limitations of RISC-V PMP and designs modular enclave components called EModules to reduce the trusted computing base (TCB) and enable enclave autonomy. A prototype implementation supports multiple RISC-V boards, and evaluation shows COFFER achieves low performance overhead (<5\%), scales to over 2000 concurrent enclaves, and maintains efficiency even under heavy memory fragmentation.

* Strengths

1.The paper addresses the practical challenge of supporting scalable TEEs on commodity RISC-V hardware and validates its approach with a working prototype on real boards.

2.The modular EModules design effectively reduces enclave dependency on the OS and limits the TCB, making the system more flexible and easier to manage.

3.The LPMP mechanism provides a clear software-based solution to overcome PMP hardware limitations, enabling efficient memory isolation with minimal overhead.
\end{generalcomment}

We sincerely thank the reviewer for the comprehensive evaluation and recognition of COFFER's contributions to scalable TEEs on commodity RISC-V platforms. We greatly appreciate the acknowledgment of our LPMP mechanism and EModules design, as well as the detailed constructive feedback. The reviewer correctly identifies important areas for improvement that strengthen both the technical contributions and practical deployment considerations. We address each concern systematically below.

%% ---------------------- 3.1: I/O Security Model ------------------------------

\begin{revcomment}
Incomplete I/O Security Model. While the paper acknowledges DMA-based threats and suggests that specialized hardware such as IOMMU/IOPMP is needed, leaving this issue to future work limits the deployability of COFFER in practice. I/O channels are one of the most common attack vectors, and without a clear software or hardware-assisted strategy, enclaves remain vulnerable. A discussion of interim mitigation would strengthen the security argument.
\end{revcomment}

\begin{revresponse}
Thank you for this important concern about I/O security. The reviewer correctly identifies that I/O channels represent a significant attack surface, and we appreciate the opportunity to provide a more comprehensive treatment of this critical aspect.

\textbf{Current I/O Security Mechanisms:} COFFER already implements several I/O security mechanisms that provide practical protection in current deployments. For memory-mapped I/O (MMIO) peripheral devices, COFFER adds special LPMP entries to protect I/O operations, ensuring that only authorized enclaves or the host OS can access specific peripheral devices. Additionally, COFFER implements a secure bulk I/O mechanism through memory ownership transfer, where the sender allocates a memory region, writes the data, and then requests the Security Monitor to transfer ownership to the receiver. Throughout this process, the principle of exclusive ownership is enforced—preventing concurrent access and ensuring data integrity.

\textbf{Secure Message Channels:} COFFER provides secure message channels for communication between enclaves and the host OS through the \texttt{SBI\_EXT\_EBI\_LISTEN\_MESSAGE} and \texttt{SBI\_EXT\_EBI\_SEND\_MESSAGE} interfaces. These channels allow controlled data exchange while maintaining enclave isolation. The Security Monitor mediates all message transfers, ensuring that only authorized communication occurs and that message buffers respect enclave memory boundaries.

\textbf{DMA Attack Mitigation Strategies:} While complete DMA protection requires specialized hardware (IOMMU/IOPMP), COFFER provides several interim mitigation strategies. COFFER can be deployed on platforms where DMA-capable devices are limited or can be controlled through administrative policies, reducing the DMA attack surface. COFFER's flexible boundary design ensures that enclave memory pools are allocated from specific physical memory regions that can be administratively configured to avoid certain DMA-capable peripherals. The autonomous enclave design further reduces attack surface by minimizing the need for extensive I/O operations, as EModules handle most system functionality within the TEE boundary.

\textbf{Hardware Integration Roadmap:} COFFER is designed with forward compatibility for emerging RISC-V security extensions. Once IOPMP becomes available, COFFER can integrate it by extending the LPMP framework to manage IOPMP entries alongside PMP entries, providing unified memory and I/O protection. Similarly, COFFER's Security Monitor can be extended to configure IOMMU page tables for enclave-specific I/O memory mappings, ensuring that DMA operations respect enclave boundaries. COFFER's modular design allows these hardware features to be added incrementally without requiring fundamental architectural changes.

\textbf{Practical Deployment Considerations:} For current deployments, COFFER addresses practical I/O security through multiple layers. COFFER's threat model explicitly acknowledges that DMA protection requires additional hardware, allowing users to make informed deployment decisions based on their specific threat landscape. The combination of memory isolation, secure message channels, and memory ownership transfer provides a foundation for secure I/O even without complete DMA protection. Applications can implement application-level encryption for sensitive data transmitted through these channels, providing additional protection layers.

We have expanded the I/O security discussion in Section VI (Security Analysis) to provide more comprehensive guidance:

\begin{changes}
\textbf{I/O security.} For MMIO peripheral devices, COFFER can add special LPMP entries for the host/enclaves to protect the I/O operations. However, there are also peripheral devices with DMA. Preventing attacks from such peripheral devices requires additional specialized hardware such as IOMMU or IOPMP, which are still under active development on RISC-V. Currently, COFFER supports enclave bulk I/O operations by memory ownership transfer, where the Security Monitor ensures exclusive access throughout the transfer process. COFFER also provides secure message channels that enable controlled communication between enclaves and the host OS while maintaining isolation. All private data should be encrypted before being sent to or received from the enclaves to provide additional protection layers. For interim deployments, COFFER's design enables deployment on platforms with limited DMA attack surface through administrative control of DMA-capable devices and careful memory region allocation. COFFER's forward-compatible design allows seamless integration of IOMMU/IOPMP when these become available on RISC-V platforms.
\end{changes}

This comprehensive approach provides practical I/O security for current RISC-V platforms while positioning COFFER for enhanced protection as specialized hardware becomes available.
\end{revresponse}

%% ---------------------- 3.2: LPMP Scalability Limitations ------------------------------

\begin{revcomment}
Scalability and Sharing Limitations. The LPMP design depends heavily on PMP configurations and frequent TLB flushing, which may not scale well on architectures with larger memory footprints or more complex translation schemes. The paper does not analyze the performance cost under such scenarios. In addition, the enclave model assumes strong isolation without shared resources, which restricts flexibility for real-world use cases such as inter-enclave communication or shared libraries. Exploring controlled sharing policies could improve practicality.
\end{revcomment}

\begin{revresponse}
% TODO: Address LPMP scalability concerns with technical analysis from Section 4.3
\end{revresponse}

%% ---------------------- 3.3: EModule Ecosystem Risks ------------------------------

\begin{revcomment}
Risks in the EModule Ecosystem. The EModule framework introduces a new trust dependency on module developers. Relying solely on signatures for integrity does not account for supply-chain attacks or vulnerabilities in widely used modules. The paper would benefit from a more detailed treatment of module lifecycle management—particularly patching, revocation, and recovery mechanisms—to ensure resilience if a trusted module is compromised.
\end{revcomment}

\begin{revresponse}
% TODO: Address EModule security concerns with details from Section 5.1 and implementation
\end{revresponse}

%% ---------------------- 3.4: Hardware Dependency Issues ------------------------------

\begin{revcomment}
Hardware Dependency of LPMP Optimizations. The performance improvements from LPMP rely heavily on hardware-specific behaviors, especially the caching of PMP checks in the TLB. This may not be present across all RISC-V implementations. Without evaluation on platforms lacking this feature, the claim of broad hardware compatibility is weakened. Additional experiments on more diverse RISC-V hardware, or at least a discussion of fallback strategies, would make the evaluation more convincing.
\end{revcomment}

\begin{revresponse}
% TODO: Address hardware dependency concerns with analysis from Section 4.3 and background
\end{revresponse}

%% ---------------------- 3.5: Limited Security Analysis ------------------------------

\begin{revcomment}
Limited Security Analysis and Trade-offs. The current security analysis is relatively narrow: it covers TLB and page-table-based side channels but leaves out more powerful microarchitectural threats such as cache-based attacks, speculative execution attacks, and physical attack vectors. This omission leaves enclaves potentially vulnerable. Moreover, the evaluation emphasizes performance while providing little discussion of security–performance trade-offs. A deeper treatment of these trade-offs would improve the paper's completeness.
\end{revcomment}

\begin{revresponse}
% TODO: Address security analysis concerns with expansion beyond current Section 6.1
\end{revresponse}
