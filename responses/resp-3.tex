\reviewer{Reviewer 3}

\begin{generalcomment}
This paper presents COFFER, a scalable and efficient Trusted Execution Environment (TEE) for commodity RISC-V platforms. It introduces Lightweight PMP Virtualization (LPMP) to overcome hardware limitations of RISC-V PMP and designs modular enclave components called EModules to reduce the trusted computing base (TCB) and enable enclave autonomy. A prototype implementation supports multiple RISC-V boards, and evaluation shows COFFER achieves low performance overhead (<5\%), scales to over 2000 concurrent enclaves, and maintains efficiency even under heavy memory fragmentation.

* Strengths

1.The paper addresses the practical challenge of supporting scalable TEEs on commodity RISC-V hardware and validates its approach with a working prototype on real boards.

2.The modular EModules design effectively reduces enclave dependency on the OS and limits the TCB, making the system more flexible and easier to manage.

3.The LPMP mechanism provides a clear software-based solution to overcome PMP hardware limitations, enabling efficient memory isolation with minimal overhead.
\end{generalcomment}

We sincerely thank the reviewer for the comprehensive evaluation and recognition of COFFER's contributions to scalable TEEs on commodity RISC-V platforms. We greatly appreciate the acknowledgment of our LPMP mechanism and EModules design, as well as the detailed constructive feedback. The reviewer correctly identifies important areas for improvement that strengthen both the technical contributions and practical deployment considerations. We address each concern systematically below.

%% --------- 3.1: I/O Security Model -------------

\begin{revcomment}
Incomplete I/O Security Model. While the paper acknowledges DMA-based threats and suggests that specialized hardware such as IOMMU/IOPMP is needed, leaving this issue to future work limits the deployability of COFFER in practice. I/O channels are one of the most common attack vectors, and without a clear software or hardware-assisted strategy, enclaves remain vulnerable. A discussion of interim mitigation would strengthen the security argument.
\end{revcomment}

\begin{revresponse}
Thank you for this important concern about I/O security. The reviewer correctly identifies that I/O channels represent a significant attack surface, and we appreciate the opportunity to provide a more comprehensive treatment of this critical aspect.

\textbf{Current I/O Security Mechanisms:} COFFER already implements several I/O security mechanisms that provide practical protection in current deployments. For memory-mapped I/O (MMIO) peripheral devices, COFFER adds special LPMP entries to protect I/O operations, ensuring that only authorized enclaves or the host OS can access specific peripheral devices. Additionally, COFFER implements a secure bulk I/O mechanism through memory ownership transfer, where the sender allocates a memory region, writes the data, and then requests the Security Monitor to transfer ownership to the receiver. Throughout this process, the principle of exclusive ownership is enforced—preventing concurrent access and ensuring data integrity.

\textbf{Secure Message Channels:} COFFER provides secure message channels for communication between enclaves and the host OS through the \texttt{SBI\_EXT\_EBI\_LISTEN\_MESSAGE} and \texttt{SBI\_EXT\_EBI\_SEND\_MESSAGE} interfaces. These channels allow controlled data exchange while maintaining enclave isolation. The Security Monitor mediates all message transfers, ensuring that only authorized communication occurs and that message buffers respect enclave memory boundaries.

\textbf{DMA Attack Mitigation Strategies:} While complete DMA protection requires specialized hardware (IOMMU/IOPMP), COFFER provides several interim mitigation strategies. COFFER can be deployed on platforms where DMA-capable devices are limited or can be controlled through administrative policies, reducing the DMA attack surface. COFFER's flexible boundary design ensures that enclave memory pools are allocated from specific physical memory regions that can be administratively configured to avoid certain DMA-capable peripherals. The autonomous enclave design further reduces attack surface by minimizing the need for extensive I/O operations, as EModules handle most system functionality within the TEE boundary.

\textbf{Hardware Integration Roadmap:} COFFER is designed with forward compatibility for emerging RISC-V security extensions. Once IOPMP becomes available, COFFER can integrate it by extending the LPMP framework to manage IOPMP entries alongside PMP entries, providing unified memory and I/O protection. Similarly, COFFER's Security Monitor can be extended to configure IOMMU page tables for enclave-specific I/O memory mappings, ensuring that DMA operations respect enclave boundaries. COFFER's modular design allows these hardware features to be added incrementally without requiring fundamental architectural changes.

\textbf{Practical Deployment Considerations:} For current deployments, COFFER addresses practical I/O security through multiple layers. COFFER's threat model explicitly acknowledges that DMA protection requires additional hardware, allowing users to make informed deployment decisions based on their specific threat landscape. The combination of memory isolation, secure message channels, and memory ownership transfer provides a foundation for secure I/O even without complete DMA protection. Applications can implement application-level encryption for sensitive data transmitted through these channels, providing additional protection layers.

We have expanded the I/O security discussion in Section VI (Security Analysis) to provide more comprehensive guidance:

\begin{changes}
\textbf{I/O security.} For MMIO peripheral devices, COFFER can add special LPMP entries for the host/enclaves to protect the I/O operations. However, there are also peripheral devices with DMA. Preventing attacks from such peripheral devices requires additional specialized hardware such as IOMMU or IOPMP, which are still under active development on RISC-V. Currently, COFFER supports enclave bulk I/O operations by memory ownership transfer, where the Security Monitor ensures exclusive access throughout the transfer process. COFFER also provides secure message channels that enable controlled communication between enclaves and the host OS while maintaining isolation. All private data should be encrypted before being sent to or received from the enclaves to provide additional protection layers. For interim deployments, COFFER's design enables deployment on platforms with limited DMA attack surface through administrative control of DMA-capable devices and careful memory region allocation. COFFER's forward-compatible design allows seamless integration of IOMMU/IOPMP when these become available on RISC-V platforms.
\end{changes}

This comprehensive approach provides practical I/O security for current RISC-V platforms while positioning COFFER for enhanced protection as specialized hardware becomes available.
\end{revresponse}

%% -------- 3.2: LPMP Scalability Limitations --------

\begin{revcomment}
Scalability and Sharing Limitations. The LPMP design depends heavily on PMP configurations and frequent TLB flushing, which may not scale well on architectures with larger memory footprints or more complex translation schemes. The paper does not analyze the performance cost under such scenarios. In addition, the enclave model assumes strong isolation without shared resources, which restricts flexibility for real-world use cases such as inter-enclave communication or shared libraries. Exploring controlled sharing policies could improve practicality.
\end{revcomment}

\begin{revresponse}
Thank you for this insightful question about LPMP scalability limitations. The reviewer correctly identifies important considerations regarding the scalability of our trap-and-emulate approach, and we appreciate the opportunity to provide a more thorough analysis of LPMP's theoretical and practical scalability bounds.

\textbf{Theoretical Scalability Analysis:} LPMP's trap-and-emulate approach has well-defined theoretical scalability limits. The primary bottleneck occurs when the working set of memory regions exceeds the available PMP entries ($N_{seg}^{active} > N_{PMP}^{max}$), triggering frequent LPMP traps. In the worst case, each memory access to a non-cached LPMP entry incurs a trap overhead of approximately 1000-2000 cycles on RISC-V platforms. However, COFFER's instruction/data split optimization significantly reduces this overhead by reserving dedicated PMP entries for instruction memory, which typically exhibits high locality. Our TLB-enhancement optimization further improves scalability by effectively extending the number of active PMP entries through TLB caching, particularly beneficial for workloads accessing larger memory footprints with reasonable spatial locality.

\textbf{Memory Footprint Scalability:} COFFER has been evaluated with enclaves up to 2GB memory size with less than 7\% performance overhead. For larger memory footprints, LPMP's performance depends on memory access patterns rather than absolute memory size. Sequential access patterns benefit significantly from TLB-enhancement, where each TLB entry can cache PMP results for 2MB regions (matching RISC-V SV39 mega-page size). Random access patterns rely more heavily on instruction/data split optimization. The key insight is that LPMP scales with the active working set of memory regions, not total enclave memory size. Large enclaves with localized memory access maintain near-native performance, while those with scattered access patterns across many regions experience higher overhead.

\textbf{TLB Flushing Impact and Mitigation:} The reviewer's concern about frequent TLB flushing is valid but mitigated by COFFER's selective TLB management. COFFER implements two TLB flushing strategies: (1) full TLB flush during enclave context switches to ensure exclusive TLB ownership, and (2) selective TLB entry invalidation during LPMP traps when TLB-enhancement is enabled. The selective approach only invalidates the specific TLB entry causing the trap, preserving other cached PMP results. This dramatically reduces TLB flush overhead compared to naive approaches that would flush the entire TLB on every LPMP configuration change. On multi-core systems, TLB flush costs are further amortized since each core maintains its own TLB state.

\textbf{Scalability with Different RISC-V Configurations:} COFFER's scalability adapts to different RISC-V memory management configurations. On platforms with more PMP entries (16+ instead of the typical 8), LPMP naturally scales better due to reduced trap frequency. For platforms without TLB caching of PMP results, COFFER gracefully degrades to rely primarily on instruction/data split optimization. The modular design allows runtime detection of platform capabilities and automatic adaptation of optimization strategies. COFFER has been tested on three different RISC-V platforms (HiFive Unmatched with SiFive U74, VisionFive 2, and D1 Nezha) with consistent scalability characteristics, demonstrating broad hardware compatibility.

\textbf{Addressing Inter-Enclave Communication Limitations:} Regarding the reviewer's concern about strong isolation limiting inter-enclave communication, COFFER provides controlled communication mechanisms while maintaining security. COFFER implements secure message channels through \texttt{SBI\_EXT\_EBI\_LISTEN\_MESSAGE} and \texttt{SBI\_EXT\_EBI\_SEND\_MESSAGE} interfaces, enabling authorized data exchange between enclaves and the host OS. For inter-enclave communication, the current design prioritizes security over flexibility by maintaining strict isolation. However, controlled sharing could be implemented through Security Monitor-mediated shared memory regions with fine-grained access control, enabling use cases like shared libraries or collaborative computing while preserving the fundamental security guarantees.

We have expanded Section IV-C (LPMP Design) to provide more comprehensive scalability analysis:

\begin{changes}
The theoretical scalability of LPMP depends on the active working set of memory regions rather than total enclave memory size. When the number of active memory segments exceeds available PMP entries ($N_{seg}^{active} > N_{PMP}^{max}$), trap frequency increases proportionally. However, the instruction/data split optimization reserves dedicated PMP entries for instruction memory, significantly reducing traps for code with high spatial locality. TLB-enhancement further extends effective PMP capacity by caching results for 2MB regions, enabling COFFER to maintain near-native performance even for large memory footprints with reasonable access locality. The selective TLB invalidation strategy minimizes flush overhead by preserving cached PMP results for non-conflicting memory regions, making LPMP practical even under memory-intensive workloads.
\end{changes}

This analysis demonstrates that while LPMP has theoretical scalability limits, the practical performance remains excellent for realistic workloads through careful optimization and adaptive hardware utilization.
\end{revresponse}

%% -------- 3.3: EModule Ecosystem Risks --------

\begin{revcomment}
Risks in the EModule Ecosystem. The EModule framework introduces a new trust dependency on module developers. Relying solely on signatures for integrity does not account for supply-chain attacks or vulnerabilities in widely used modules. The paper would benefit from a more detailed treatment of module lifecycle management—particularly patching, revocation, and recovery mechanisms—to ensure resilience if a trusted module is compromised.
\end{revcomment}

\begin{revresponse}
Thank you for this critical concern about EModule ecosystem security. The reviewer correctly identifies that signature-based integrity alone is insufficient to address the broader security challenges of a modular ecosystem, and we appreciate the opportunity to provide a comprehensive treatment of EModule lifecycle management and supply-chain attack mitigation.

\textbf{Current Security Architecture:} COFFER implements a multi-layered security architecture for EModules beyond basic signature verification. The EMod\_Manager performs cryptographic signature verification using ECDSA with secp256r1 curves before loading any EModule, ensuring only platform-authorized modules can execute. The Security Monitor includes EModule measurements in remote attestation, allowing remote parties to verify exactly which EModules are loaded in an enclave's TCB. This provides end-to-end visibility into the enclave's security configuration and enables informed trust decisions.

**Supply-Chain Attack Mitigation:** COFFER addresses supply-chain risks through several architectural and operational mechanisms. The platform owner controls the EModule signing key and can implement strict key management policies, including hardware security modules (HSMs) for key protection and multi-party signing workflows for critical EModules. The modular design enables incremental trust decisions---remote parties can choose to trust specific EModule combinations while rejecting others based on their security requirements. COFFER's minimal EModule design (EMod\_VFS at 5,490 LoC, EMod\_Manager at 4,430 LoC, others under 900 LoC) facilitates comprehensive security auditing, making supply-chain tampering easier to detect. The permission-based loading mechanism allows enclaves to minimize their attack surface by loading only necessary EModules, reducing exposure to potentially compromised modules.

\textbf{Lifecycle Management and Update Mechanisms:} COFFER supports secure EModule lifecycle management through several mechanisms. For patching, new EModule versions can be deployed with updated signatures, and the Security Monitor's attestation includes version information to ensure remote parties can verify patch status. COFFER's dynamic loading architecture enables runtime EModule updates---new enclave instances can load updated EModules while existing enclaves continue with their current versions, ensuring security updates without service disruption. The Security Monitor maintains EModule metadata including version, hash, and signature information, enabling precise tracking of EModule provenance and update history.

\textbf{Revocation and Recovery Mechanisms:} COFFER implements comprehensive revocation capabilities for compromised EModules. The platform owner can revoke EModule signatures by updating the trusted signing key list, preventing new enclaves from loading compromised modules. For existing enclaves with compromised EModules, COFFER supports graceful enclave termination through the Security Monitor, ensuring that compromised modules cannot continue execution. Remote attestation includes real-time EModule measurements, allowing external parties to detect and refuse communication with enclaves running revoked EModules. COFFER's isolation guarantees ensure that even if an EModule is compromised within one enclave, it cannot affect other enclaves or the host system.

\textbf{Enhanced Trust and Verification Framework:} Beyond signatures, COFFER enables additional verification mechanisms. The deterministic build system ensures reproducible EModule binaries, enabling community verification of EModule integrity. The Security Monitor can be extended to support policy-based EModule loading, where platform administrators define rules about acceptable EModule combinations and versions. COFFER's design enables integration with emerging technologies like code transparency frameworks and binary provenance systems, providing cryptographic proof of EModule build integrity from source to deployment.

\textbf{Compartmentalization and Damage Limitation:} COFFER's architecture limits the impact of compromised EModules through strong isolation boundaries. While EModules within an enclave share the S-Mode protection domain for efficiency, each enclave runs completely isolated EModule instances---a compromised EModule in one enclave cannot access or affect EModules in other enclaves. The Security Monitor maintains strict memory isolation, preventing compromised EModules from accessing host system resources or other enclave memory. COFFER's autonomous enclave design reduces the need for complex EModule interactions, limiting the attack surface and potential for privilege escalation.

We have expanded Section VI (Security Analysis) to provide comprehensive coverage of EModule ecosystem security:

\begin{changes}
\textbf{EModule Ecosystem Security.} COFFER addresses EModule ecosystem risks through comprehensive lifecycle management beyond signature verification. The platform owner controls EModule signing keys and can implement multi-party signing workflows and hardware security modules for enhanced key protection. Remote attestation includes EModule measurements, enabling remote parties to verify exact EModule configurations and make informed trust decisions. COFFER supports secure EModule updates through versioned signatures and dynamic loading, enabling security patches without service disruption. Revocation mechanisms allow platform owners to prevent loading of compromised EModules and terminate enclaves running revoked modules. The minimal EModule design (EMod\_VFS at 5,490 LoC, EMod\_Manager at 4,430 LoC, others under 900 LoC) facilitates comprehensive security auditing, while strong inter-enclave isolation ensures that compromised EModules cannot affect other enclaves or the host system. Integration with reproducible builds and code transparency frameworks provides additional verification layers beyond signature-based integrity.
\end{changes}

This comprehensive approach addresses the full spectrum of EModule ecosystem risks while maintaining the practical benefits of modular enclave construction and autonomous operation.
\end{revresponse}

%% -------- 3.4: Hardware Dependency Issues --------

\begin{revcomment}
Hardware Dependency of LPMP Optimizations. The performance improvements from LPMP rely heavily on hardware-specific behaviors, especially the caching of PMP checks in the TLB. This may not be present across all RISC-V implementations. Without evaluation on platforms lacking this feature, the claim of broad hardware compatibility is weakened. Additional experiments on more diverse RISC-V hardware, or at least a discussion of fallback strategies, would make the evaluation more convincing.
\end{revcomment}

\begin{revresponse}
% TODO: Address hardware dependency concerns with analysis from Section 4.3 and background
\end{revresponse}

%% -------- 3.5: Limited Security Analysis -----------

\begin{revcomment}
Limited Security Analysis and Trade-offs. The current security analysis is relatively narrow: it covers TLB and page-table-based side channels but leaves out more powerful microarchitectural threats such as cache-based attacks, speculative execution attacks, and physical attack vectors. This omission leaves enclaves potentially vulnerable. Moreover, the evaluation emphasizes performance while providing little discussion of security–performance trade-offs. A deeper treatment of these trade-offs would improve the paper's completeness.
\end{revcomment}

\begin{revresponse}
% TODO: Address security analysis concerns with expansion beyond current Section 6.1
\end{revresponse}
