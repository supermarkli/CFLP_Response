% Reviewer 1(中文)
\reviewer

\begin{revcomment}{对论文核心结论的总结}
论文研究多种机器学习算法在 TEE 内采用不同聚合方法(不同 TEE 方案、HE 与 MPC)所能获取的精度和时间问题。所得结论是:在保持基线精度的前提下,基于 TDX 的可信执行环境方案仅增加约 11\% 时延即可提供硬件级保护,综合表现优于 SGX、HE 与 MPC;HE 虽具有可形式化验证的安全性,但计算与通信开销分别提升至基线的数百倍,MPC 则在时间与通信开销间取得折中。
\end{revcomment}
\begin{revresponse}
感谢审稿专家对我们工作的精准提炼。我们完全同意您的总结,这也是我们工作的核心发现。
\end{revresponse}

\begin{revcomment}{对实验结论符合预期的认可}
该结论是符合预期的。在没有实验的情况下, m 虽然不能获得准确的数量对比关系,但仍然可推出各个方案所花费的时间大小关系。
\end{revcomment}
\begin{revresponse}
感谢审稿专家对我们工作的认可。我们在表4中提供了各个方案的平均训练时长,感谢您的建议。
\end{revresponse}

\begin{concludingresponse}[]
感谢审稿专家的宝贵意见与建议,我们已据此完善正文相关章节与数据呈现。
\end{concludingresponse}
