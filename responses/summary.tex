\section{修改概述}

\indent \indent
我们已根据审稿意见对论文进行了系统性修改与补充,主要包括:
\begin{itemize}
    \item \textbf{重构与澄清核心概念:} 新增“威胁模型与安全假设”章节,重写了 TEE 基本思想、联邦模拟平台与实验平台关系等关键定义,确保了文章核心概念的清晰与准确。
    \item \textbf{补充系统设计细节:} 详细阐述了平台的通信机制 (gRPC)、身份验证、数据加密与安全闭环设计,回应了关于系统可用性、安全性和部署的疑问。
    \item \textbf{深化实验分析:} 不再局限于数据对比,而是从算法机理、模型架构等维度深入剖析了实验结果差异的根本原因,提升了分析的深度。
    \item \textbf{增强图表可读性:} 重绘了图4(平台时序图)与图6(性能对比图),优化了排版与字体,确保了图表的清晰易读。
    \item \textbf{增补安全性分析:} 新增“安全性分析”章节,系统性论述了平台如何在计算、传输和存储全周期内保障机密性与完整性。
\end{itemize}

为便于审阅,我们在修改稿中以【红色】标注新增或改写的内容。