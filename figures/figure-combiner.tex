\documentclass{article}
\usepackage{xeCJK} % 引入 xeCJK 宏包
\setCJKmainfont{WenQuanYi Zen Hei} % 设置一个你系统里有的中文字体
\usepackage{graphicx}
\usepackage{subcaption}
\usepackage[margin=1in]{geometry} % 设置页边距,让图片有足够空间

\begin{document}
\pagestyle{empty} % 添加此行来禁用页码

\begin{figure}[p] % 使用 [p] 参数让图形单独占一页
    \centering
    \captionsetup{labelformat=empty} % 不显示 "Figure 1:" 这样的主标题

    % --- 第一行 ---
    \begin{subfigure}[b]{0.48\textwidth}
        \centering
        \includegraphics[width=\linewidth]{fig-2.pdf} % 替换为您的文件路径
        \caption{客户端注册与启动同步}
        \label{fig:sub_a}
    \end{subfigure}
    \hfill % 在两个子图之间添加弹性空间
    \begin{subfigure}[b]{0.48\textwidth}
        \centering
        \includegraphics[width=\linewidth]{fig-3.pdf} % 替换为您的文件路径
        \caption{本地训练与安全聚合}
        \label{fig:sub_b}
    \end{subfigure}

    \vspace{0.5cm} % 在两行之间添加一些垂直间距

    % --- 第二行 ---
    \begin{subfigure}[b]{0.48\textwidth}
        \centering
        \includegraphics[width=\linewidth]{fig-4.pdf} % 替换为您的文件路径
        \caption{全局模型更新与迭代}
        \label{fig:sub_c}
    \end{subfigure}

\end{figure}

\end{document}